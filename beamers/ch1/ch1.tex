% =========================================================================
%
%
% =========================================================================
\documentclass[aspectratio=1610]{beamer}

% ========================= Theme =========================================
\usetheme{CambridgeUS}
\usecolortheme{default}

% ========================= Packages ======================================
\usepackage{calc}
\usepackage{tikz}
\usetikzlibrary{arrows,
                arrows.meta,
                calc,
		chains,
                quotes,
                positioning,
		shapes,
                shapes.geometric}
\usepackage{graphicx}
\usepackage{graphics}
\usepackage{pgfplots}
\pgfplotsset{width=7cm,compat=1.17}

% ========================= Infor on authors ==============================
\title{Intro to Python --- SMM692}
\subtitle{Organization of the Module}
\author{Simone Santoni}
\institute{Bayes Business School}
\date{MSc Pre-Course Series}

% ============================ Colors =====================================
\definecolor{base_c}{rgb}{0.6,0,0}
\definecolor{comp_c}{rgb}{0.09803921568627451, 0.6901960784313725, 0.7529411764705882}
\definecolor{tri_1}{rgb}{0.09803921568627451, 0.7686274509803922, 0.19215686274509805}
\definecolor{tri_2}{rgb}{0.19215686274509805, 0.09803921568627451, 0.7686274509803922}

% ========================= TOC  ==========================================
\AtBeginSubsection[]
{
    \begin{frame}
        \frametitle{Outline}
        \tableofcontents[currentsection,currentsubsection]
    \end{frame}
}

% ========================= Document  ====================================
\begin{document}

\begin{frame}
	\titlepage
\end{frame}

\begin{frame}{Outline}
	\tableofcontents
\end{frame}

% ------------------------- Background -----------------------------------

\section{Background}

\subsection{Justification for SMM692}

\begin{frame}{Why Python?}
	\begin{columns}
		\begin{column}{0.5\textwidth}
			\begin{itemize}
				\item Python is a general-purpose, high-level programming language
				\begin{itemize}
					\item Traditionally, Python was used for developing desktop and web applications
				\end{itemize}
				\item So, why should business analytics (BA) professionals learn it?
				\begin{itemize}
				\item Python is the center of one of the richest ecosystems of data science modules
				\item Second, for a large number of developers and users `Python' and `data science' are largely overlapping categories  (YR 2014 $\rightarrow$)
			        \end{itemize}
		        \end{itemize}
               \end{column}
	       \begin{column}{0.5\textwidth}
		\begin{figure}
			\centering
			\begin{tikzpicture}[scale=1]
				\begin{axis}[
					title = Interest Over Time (Worldwide),
					xtick={0, 47, 95, 143},
					xticklabels={2010, 2014, 2018, 2022},
					xlabel = Time,
					ylabel = Google Trends Index,
					legend pos=north west
					]
				    \addplot[
					scatter,only marks,scatter src=explicit symbolic,
					scatter/classes={
					    python={mark=*,draw=base_c, fill=base_c!20},
					    r={mark=*,draw=tri_1, fill=tri_1!20},
					    julia={mark=*,draw=tri_2,fill=tri_2!20}
					}
				    ]
				    table[x=,y=y,meta=label]{data/google_trends.dat};
				    \legend{Python, R, Julia}
				\end{axis}
				\end{tikzpicture}
		\end{figure}
	       \end{column}
        \end{columns}
\end{frame}

\begin{frame}{Why a Pre-Course Module on Python in a BA Post-Grad Course?}
	There are at least three arguments:
	\begin{itemize}
		\item Modern BA curricula increasingly depart from a spreadsheet-based approach to teaching and learning
		\item The Bayes BA program builds heavily on the Python programming language
		\begin{itemize}
			\item There are \emph{four core modules} based on Python: Data Visualization, Deep Learning, Network Analytics, Value Creation in Digital Settings
			\item Plus \emph{two elective modules}: Applied Machine Learning and Applied Natural Language Processing
		\end{itemize}
		\item You may want to get up and running with Python in advance, that is, before the start of Term I
	\end{itemize}
\end{frame}

% ------------------------- Scope ----------------------------------------
\subsection{Scope of SMM692}

\begin{frame}[t]{Python for What?}
\begin{columns}[t]
	\begin{column}{0.5\textwidth}
		\textcolor{tri_1}{Will SMM692 cover the entire spectrum of the Python language?}

		\vspace{1em}

		\textcolor{tri_1}{Short answer: NO}

		\vspace{1em}

		\begin{itemize}
			\item That would take substantial time --- Python has an extensive language reference!
			\item Some aspects of the Python language reference have limited added value for BA professionals, who are closer to Python users than developers
		\end{itemize}
	\end{column}
	\begin{column}{0.5\textwidth}
		\textcolor{tri_2}{So, which parts of the Python language will I learn?}

		\vspace{1em}

		\textcolor{tri_2}{Short answer: the basics of the language}

		\vspace{1em}

		\begin{itemize}
			\item Variables
			\item Object
			\item Some built-in functions, string-, list-, dictionary-, and set-methods
		\end{itemize}
	\end{column}
\end{columns}
\end{frame}

\begin{frame}[c]{Python in the Business Analytics Sphere}

	\begin{columns}[c]
		\begin{column}{0.5\textwidth}

			\begin{itemize}
				\item Learning the basics of the Python language will put you in the position to carry out technical and scentific computation tasks --- namely, the essence of data science X BA
				\item Talking about the technical and scientific compuation: there are two foundational modules that support stastitical analysis, Viz, ML, and DL: NumPy and Pandas. These two modules are the focus of SMM692's second part
			\end{itemize}
		\end{column}
		\begin{column}{0.5\textwidth}
			\centering
			\includegraphics[width=0.25\textwidth]{images/numpy.png}

			\includegraphics[width=0.4\textwidth]{images/pandas.png}
		\end{column}
	\end{columns}
\end{frame}

\begin{frame}{SMM692 Foci}
	\begin{enumerate}
		\item Getting started with Python
		\item Python language fundamentals
		\item Python objects and methods
		\item Technical and scientific computation with NumPy and SciPy
		\item Data management with Pandas
	\end{enumerate}
\end{frame}

% ------------------------- Teaching and learning activities --------------
\section{Learning/Teaching Actitivies}

\subsection{Phylosophy Behind SMM692}

\begin{frame}{The Two Pillars of the Module}
\centering
	\LARGE LESS IS MORE

	\Large Focus on few core Python notions at a time and practice them

\vspace{1.5em}

	\LARGE LEARN BY DOING

	\Large The best way to learn Python is by addressing concrete problems

\end{frame}

\begin{frame}{I'm a Programming Newbie: How Do I Learn Python?}

	\begin{columns}
		\begin{column}{0.5\textwidth}
			Here is an approach the proved to be effcective according to the pedagogical literature on learnning programming languages:
			\begin{itemize}
				\item Step 1: Focus on \emph{few core notions} at a time (e.g., `string methods')
				\item Step 2: Learn the selected core notions `pen \& paper'  (stay away from the computer!)
				\item Step 3: Open a Python session and practice the few core notions 
				\item Step 4: Critically self-assess your learning (if your code is not working, Python will force you to understand what's wrong!)
			\end{itemize}
		\end{column}
		\begin{column}{0.5\textwidth}
			\centering
			\includegraphics[width=0.75\textwidth]{images/python_newbies.png}
		\end{column}
	\end{columns}
\end{frame}

\subsection{Assessment}

\begin{frame}{Self-Assess your Learning, Python Topic by Topic}
	\tikzstyle{decision} = [
		diamond,
		draw,  
		text width=8em,
		text centered, 
		node distance=2cm, 
		inner sep=0pt
		]
	\tikzstyle{block} = [
		rectangle, 
		draw,
		text width=8em,
		text centered,
		rounded corners,
		minimum height=2em
		]
	\tikzstyle{arrow} = [
		draw,
		-latex',
		rounded corners
		]
	\tikzstyle{line} = [
		draw
		]
	\tikzstyle{invisible4} = [
		rectangle
		]
	\tikzstyle{invisible5} = [
		rectangle
		]
	\tikzstyle{circular} = [
		draw,
		circle,
		radius=1.0cm,
		]

\begin{figure}[h]
    \centering
    \begin{tikzpicture}[
	scale=.55, every node/.style={scale=0.55},
	node distance = 2cm,
	auto,
	->,
	]
        %% nodes
	\node [block] (start) {\small Start / open the learning circle};
        \node [block, right = 4.75cm of start] (stop) {\small Stop / close the learning circle};
        \node [block, below of = start] (hv) {\small Watch the hook videos};
        \node [block, below of = hv] (notes) {\small Study the notes, including programming notions and examples};
        \node [block, below of = notes] (quiz) {\small Take the quiz};
	\node [decision, below=0.5cm of quiz] (decision_1) {Is your learning satisfying?};
        \node [block, right=1.25cm of decision_1] (ps) {\small Analyze the problem set};
        \node [decision, right=1.25cm of ps] (decision_2) {\small Can you address the problem set?};
        \node [decision, right=1.25cm of decision_2] (decision_3) {\small Why not?};
	% \node [block, above = 1 cm of decision_2] (tr) {\small Analyze the traceback of the error};
	%% arrows
	\path[]
	  (start) edge node [] {} (hv)
	  (hv) edge node [] {} (notes)
	  (notes) edge node [] {} (quiz)
	  (quiz) edge node [] {} (decision_1)
	  (decision_1) edge node [] {Yes} (ps)
	  (decision_1.west) edge[bend left=90] node [] {No} (notes.west)
	  (ps) edge node[] {} (decision_2)
	  (decision_2.north) edge[] node [] {Yes} (stop.south)
	  (decision_2) edge node [] {No} (decision_3)
	  (decision_3.south) edge[bend left=45] node [] {Problem set is unclear} (ps.south)
	  (decision_3.north) edge[bend right=14] node [] {Errors in my code} (notes.east);
	\end{tikzpicture}
\end{figure}
\end{frame}

\end{document}